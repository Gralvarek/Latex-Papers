% Title: Block diagram of Third order noise shaper in Compact Disc Players
% Author: Ramón Jaramillo
\documentclass{standalone}
\usepackage{babel}
\usepackage{german}
\usepackage{textcomp}
\usepackage{tikz}
\usetikzlibrary{shapes,arrows}
\begin{document}
% Definition of blocks:
\tikzset{
block/.style    = {draw, thick, rectangle, minimum height = 3em,
  minimum width = 3em},
sum/.style      = {draw, circle, node distance = 2cm, inner sep = 0pt,
minimum size = 0.5cm}, % Adder
input/.style    = {coordinate}, % Input
output/.style   = {coordinate} % Output
}
% Defining string as labels of certain blocks.
\newcommand{\suma}{\Large$+$}
\newcommand{\inte}{$\displaystyle \int$}
\newcommand{\derv}{\huge$\frac{d}{dt}$}
\begin{tikzpicture}[auto, thick, node distance=2.25cm, >=triangle 45]
\draw
% Drawing the blocks of first filter :
    node at (0,0) [right=0mm]{}
    node [input] (input1) {} 
    node [sum, right of=input1] (suma1) {\suma}
    node [right of=suma1] (branch1) {}
    node [block, right of=branch1] (Ki) {$\displaystyle Ki\cdot\int_{0}^{t} e(t)\, dt$}
    node [block, above of=Ki] (Kp) {$Kp \cdot e(t)$}
    node [block, below of=Ki] (Kd) {$Kd\cdot\frac{d}{dt} e(t)$}
    node [sum, right of=Ki] (suma2) {\suma}
    node [block, right of=suma2] (plant) {Regelstrecke}
    node [right of=plant] (branch2) {}
    node [output, right of=branch2] (output1) {}
    node [below of=Kd] (branch3) {}
;

% Joining blocks. 
% Commands \draw with options like [->] must be written individually
\draw[->] (input1) -- node {$x(t)$} (suma1);
\draw[-] (suma1) -- node {$e(t)$} (branch1.center);
\draw[->] (branch1.center) -- (Ki);
\draw[->] (branch1.center) |- (Kp);
\draw[->] (branch1.center) |- (Kd);
\draw[->] (Ki) -- (suma2);
\draw[->] (Kp) -| (suma2);
\draw[->] (Kd) -| (suma2);
\draw[-] (suma2) -- node {\(u(t)\)} (plant);
\draw[-] (plant) -- (branch2.center);
\draw[-] (branch2.center) |- (branch3.center);
\draw[->] (branch2.center) -- node {$y(t)$} (output1);
\draw[->] (branch3.center) -| node[pos=0.97, xshift=-5] {\huge $-$} (suma1);

\filldraw
    (branch1) circle (2pt)
    (branch2) circle (2pt)
;
\end{tikzpicture}
\end{document}