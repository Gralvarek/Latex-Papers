Die Implementation einer geregelten Geradeausfahrt eines Roboters braucht ein paar Schritte. Zuerst muss ein funktionierdes Modell des Roboters gefunden werden. Danach muss irgendeine Querabweichung ber"ucksichtigt werden. Am Ende kann ein Regler f"ur das System gefunden werden. 

\subsection{Linearisierung in $X_2$-Richtung}

Die folgende Modellgleichung eines einachsigen Roboters wird gegeben:
\begin{equation} \label{eq:model}
    \overrightarrow{f}(\overrightarrow{x}, \overrightarrow{u}) =
    \begin{bmatrix*}
        \dot{x}_1 \\
        \dot{x}_2 \\
        \dot{x}_3
    \end{bmatrix*}
    =
    \begin{bmatrix*}
        \sin(x_3) & 0 \\
        \cos(x_3) & 0 \\
        0 & 1
    \end{bmatrix*}
    \begin{bmatrix*}
        v \\
        \omega
    \end{bmatrix*}
\end{equation}.

Wegen des nichtlinearen Forms der Modellgleichung ist es schwierig das Systemverhalten des Modells zu beobachten. Deshalb wird sie in einem Arbeitspunkt linearisiert und dann wird dieses Modells betrachtet. Im Folge ist die Herleitung des linearisierten Modells:

\begin{equation}
    \overrightarrow{\widetilde{f}}(\overrightarrow{\widetilde{x}}, \overrightarrow{\widetilde{u}}) =
    \begin{bmatrix*}
        \dot{x}_1 \\
        \dot{x}_2 \\
        \dot{x}_3
    \end{bmatrix*}
    =
    \begin{bmatrix*}
        0 & 0 & \cos(x_{30})v_0 \\
        0 & 0 & -\sin(x_{30})v_0 \\
        0 & 0 & 0
    \end{bmatrix*}
    \begin{bmatrix*}
        \widetilde{x_1} \\
        \widetilde{x_2} \\
        \widetilde{x_3} 
    \end{bmatrix*}
    +
    \begin{bmatrix*}
        \sin(x_{30}) & 0 \\
        \cos(x_{30}) & 0 \\
        0 & 1
    \end{bmatrix*}
    \begin{bmatrix*}
        \widetilde{v} \\
        \widetilde{\omega}
    \end{bmatrix*}
\end{equation}.

Mit
\begin{equation*}
    \overrightarrow{\widetilde{x}} =
    \begin{bmatrix*} 
        x_1 - x_{10} \\
        x_2 - x_{20} \\
        x_3 - x_{30} 
    \end{bmatrix*}
    =
    \begin{bmatrix*} 
        \widetilde{x_1} \\
        \widetilde{x_2} \\
        \widetilde{x_3} 
    \end{bmatrix*},
    \overrightarrow{\widetilde{u}} =
    \begin{bmatrix*}
        v - v_0\\
        \omega - \omega_0
    \end{bmatrix*}
    =
    \begin{bmatrix*}
        \widetilde{v} \\
        \widetilde{\omega}
    \end{bmatrix*}.
\end{equation*}

Bei einem Geradefahren in dem \(X_2\)-Richtung wird die \(X_1\)- und \(X_3\)-Richtungen ignoriert. Deswegen wird die Parameter der Funktion auf die folgenden Werte gesetzt:


\begin{equation}
    \left. \widetilde{f}(\overrightarrow{\widetilde{x}}, \overrightarrow{\widetilde{u}})\right\rvert_{x_2} = \dot{x}_2 = \widetilde{v}.
\end{equation}

\subsection{Querabweichung}

In der reellen Welt kann der Roboter nicht perfekt gerade fahren. Deshalb muss die Querabweichung der Fahrt ber"ucksichtigt werden. Die Querabweichung \(\overrightarrow{e}\) ist als der Abstand zwischen dem geplanten Zielpunkt \(\overrightarrow{r}\) und der jetztigen Endposition \(\overrightarrow{x}\) des Roboters definiert, wie in der folgenden Gleichung gezeigt wird:

\begin{equation*}
    \overrightarrow{e} = \overrightarrow{r} - \overrightarrow{x}.
\end{equation*} \\

Der Zielpunkt \(\overrightarrow{r}\) ist bei der Geradeausfahrt nur die \(X_2\)-Komponent des Positionsvectors.  Die Gleichung im Vector-Form wird in der folgenden Gleichung gezeigt:

\begin{equation*}
    \begin{bmatrix}
        0 \\
        x_2 \\
        0
    \end{bmatrix}
    -
    \begin{bmatrix}
        x_1 \\
        x_2 \\
        0
    \end{bmatrix}
    =
    \begin{bmatrix}
        -x_1 \\
        0 \\
        0
    \end{bmatrix}
\end{equation*}.

Die obere Gleichung l"asst sich mit der folgenden eindimensionalen Gleichung vereinfachen:

\begin{equation}
    e(t) = -x_1(t).
\end{equation}

Bei der ersten Ableitung der oberen Gleichung nach der Zeit wird der Zussamenhang zwischen der Querabweichung und dem Modell gezeigt. Der Zussamenhang ist in der folgenden Gleichung gezeigt:

\begin{equation}
    \dot{e} = -\dot{x}_1.
\end{equation}

\subsection{Wurzelortskurven}    

\subsection{Implementation}

Zur Zeit des Abschliessens dieser Seminararbeit k"onnte eine funktionierende Implementation der geregelten Geradeausfahrt nicht gefunden werden. Mehr Information steht im Kapitel 6.\\




